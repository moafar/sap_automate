\documentclass{article}
\usepackage[utf8]{inputenc}
\usepackage[spanish]{babel}
\usepackage{hyperref}
\usepackage{listings}
\usepackage{xcolor}

\title{Arquitectura de la Aplicación de Scripts SAP}
\author{Documentación del Proyecto}
\date{\today}

\begin{document}

\maketitle

\section{Introducción}
Este documento describe la arquitectura de la aplicación de automatización de scripts SAP. La aplicación está organizada en capas modulares para separar responsabilidades y facilitar el mantenimiento.

\section{Capas de la Aplicación}

\subsection{1. Capa de Configuración (\texttt{config/})}
\textbf{Propósito:} Centralizar parámetros y secretos para evitar valores codificados en el código fuente.

\begin{itemize}
    \item \textbf{Archivos clave:}
    \begin{itemize}
        \item \texttt{settings.yaml}: Configuración general de la aplicación.
        \item \texttt{secrets.yaml}: Credenciales y datos sensibles (no versionado).
        \item \texttt{field\_mappings.yaml}: Mapeo de campos para transformaciones de datos.
    \end{itemize}
\end{itemize}

\subsection{2. Capa Core (\texttt{src/core/})}
\textbf{Propósito:} Gestionar la interacción directa y de bajo nivel con SAP GUI Scripting.

\begin{itemize}
    \item \textbf{Componentes:}
    \begin{itemize}
        \item \texttt{sap\_connection.py}: Maneja la conexión, sesión y el proceso de login con SAP GUI.
        \item \texttt{sap\_utils.py}: Contiene funciones genéricas de navegación y manipulación de la interfaz de usuario de SAP.
    \end{itemize}
\end{itemize}

\subsection{3. Capa de Utilidades (\texttt{src/utils/})}
\textbf{Propósito:} Proveer herramientas transversales de soporte que son utilizadas por otras capas.

\begin{itemize}
    \item \textbf{Componentes:}
    \begin{itemize}
        \item \texttt{logger.py}: Sistema centralizado de registro de eventos (logs).
        \item \texttt{credential\_manager.py}: Gestión segura de credenciales y acceso a secretos.
        \item \texttt{field\_mapper.py}: Lógica para transformar y normalizar datos.
    \end{itemize}
\end{itemize}

\subsection{4. Capa de Scripts / Negocio (\texttt{src/scripts/})}
\textbf{Propósito:} Implementar la lógica de negocio específica para cada tarea de automatización.

\begin{itemize}
    \item \textbf{Ejemplo:}
    \begin{itemize}
        \item \texttt{export\_invoice.py}: Utiliza las capas inferiores (Core y Utils) para ejecutar el flujo específico de exportación de facturas.
    \end{itemize}
\end{itemize}

\subsection{5. Punto de Entrada (\texttt{main.py})}
\textbf{Propósito:} Orquestador principal de la aplicación. Se encarga de inicializar componentes y ejecutar los scripts seleccionados según los argumentos o configuración.

\section{Conclusión}
Esta estructura jerárquica facilita la escalabilidad y el mantenimiento. Los cambios en la conexión a SAP se aíslan en la capa Core, mientras que las nuevas automatizaciones se añaden simplemente como nuevos scripts en la capa de Negocio, reutilizando la infraestructura existente.

\end{document}
